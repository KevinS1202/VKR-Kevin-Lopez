\newsection
\section{Анализ предметной области}
\subsection{Характеристика медицинской лаборатории и ее деятельности}

Использование технологий в здравоохранении становится все более распространенным, и веб-сайт может облегчить доступ к информации и общение между сотрудниками лаборатории, врачами и пациентами. Среди основных целей внедрения веб-сайта в медицинской лаборатории - улучшение качества обслуживания пациентов, повышение эффективности процессов и снижение затрат.

Использованная методология включает исследование вторичных источников и интервью с сотрудниками медицинской лаборатории, врачами и пациентами. Было установлено, что пользователи сайта медицинской лаборатории в основном ищут информацию о лабораторных услугах и процессах, результатах анализов и лечении.

Полученные результаты показали, что, хотя большинство пользователей медицинской лаборатории знают о существовании веб-сайта, его использование относительно невелико. Кроме того, были выявлены некоторые препятствия для надлежащего использования веб-сайта, такие как недостаточная подготовка персонала и трудности с доступом к сайту с мобильных устройств.

Реализация веб-сайта будет осуществляться с использованием HTML, PHP и CSS. Эти технологии будут использованы для разработки структуры и внешнего вида сайта, а также для программирования его логики и функциональности.

В медицинской лаборатории должны быть предусмотрены меры по контролю качества и безопасности процессов, оборудование и материалы, необходимые для проведения анализов по лабораторным специальностям, а также управление и контроль со стороны обученного и квалифицированного персонала. Кроме того, она имеет стандартизированные и обновленные процедуры по сбору, хранению, анализу образцов и выдаче результатов. Она также соответствует санитарным нормам и правилам и обеспечивает эффективную связь с пользователями, как в процессе отбора проб, так и при предоставлении результатов.

Деятельность медицинской лаборатории ``KLLABORATORY'' зависит от ее специализации, но чаще всего включает в себя сбор биологических образцов у пациентов, их обработку и анализ, интерпретацию результатов и выдачу отчетов вместе с назначением лечения пациента. Наиболее распространенные анализы, выполняемые в медицинской лаборатории, включают анализы крови, мочи, кала, биологических жидкостей и тканей, а также генетические анализы. Кроме того, медицинская лаборатория может предложить специализированные медицинские консультации, генетическое консультирование и раннее выявление заболеваний, назначение и лечение. Она осуществляет сбор и анализ биологических образцов с целью внести вклад в изучение, профилактику, диагностику и лечение заболеваний у пациентов.

После анализа функций и услуг медицинской лаборатории было отмечено, что лаборатория имеет следующие специализации:

\begin{itemize}
	\item Гематология.
	\item Иммунология.
	\item Aнализ мочи.
	\item Aнализ кала.
	\item Серология.
	\item Микробиология.
\end{itemize}

Вся визуализация и интерпретация результатов, полученных в результате процессов, проведенных на пациентах, будет осуществляться только врачами, которые считаются незаметным и обученным персоналом.

Создание веб-сайта поможет медицинской лаборатории по целому ряду направлений, например:
\begin{itemize}
	\item Предоставление эффективных услуг стационарным пациентам.
	\item Облегчить общение с пациентами, позволив врачам получать доступ к их медицинским картам, получать информацию о процедурах и результатах, а также доступ к их лечению и личной информации о пациенте.
	\item Улучшить доступ к соответствующей информации, процедурам отбора проб и данным результатов медицинских лабораторных исследований.
	\item Обеспечить единую медицинскую карту для каждого пациента и удобный поиск результатов, полученных у врачей.
	\item Позволяют загружать важные формы и информацию, связанную с медицинским обслуживанием, такие как медицинские карты, протоколы анализов и формы согласия, что повышает эффективность управления информацией.
	\item Избегая потребления и использования бумаги, помогает обеспечить безопасность и конфиденциальность записей каждого пациента, зарегистрированных в медицинской лаборатории.
\end{itemize}

При соединении веб-сайта с медицинской лабораторией рассматривался процесс интеграции записей и результатов, чтобы лаборатория могла более эффективно и результативно отправлять и получать информацию от пациентов и врачей.

Веб-сайт лаборатории правильно разработан и настроен для интеграции с системами лаборатории.

Использование веб-сайта в медицинской лаборатории может иметь большое значение для улучшения коммуникации и доступа к информации для пациентов и сотрудников лаборатории. Имея веб-сайт, медицинская лаборатория может предоставить информацию о предлагаемых услугах, процедурах и других важных аспектах. Это позволяет повысить эффективность и удобство работы сотрудников медицинской лаборатории.

\subsection{Язык программирования HTML и его классификация}

HTML считается языком, используемым для создания веб-страниц и веб-приложений, который изначально был разработан для описания содержания гипертекстовых документов. Со временем он эволюционировал и стал включать в себя широкий спектр функций, таких как возможность добавления мультимедиа и интерактивных форм. Хотя HTML не является языком программирования как таковым, он играет важную роль в развитии современного Интернета и используется веб-программистами во всем мире.

В отличие от полноценных языков программирования, таких как Python или Java, HTML не способен самостоятельно программировать сложные действия и требует помощи языков программирования на стороне сервера. HTML принадлежит к семейству языков разметки SGML, и его синтаксис похож на синтаксис других языков разметки, таких как XML и XHTML. Со временем HTML эволюционировал и стал включать в себя мультимедийные функции и взаимодействие через формы.

HTML относится к категории языков разметки, наряду с другими языками, такими как XML (расширяемый язык разметки) и SGML (стандартный обобщенный язык разметки). В целом, классификация языков программирования основывается на нескольких критериях, таких как уровень абстракции, парадигма программирования и конкретное назначение языков.Однако важно отметить, что использование HTML на веб-странице медицинской лаборатории необходимо для предоставления информации о предлагаемых услугах, а также для повышения доступности и эффективности веб-сайта.

В случае медицинской лаборатории классификация HTML будет зависеть от его структуры и содержания. Вот несколько возможностей:

\begin{enumerate}
	\item Домашняя страница: получение общей информации о лаборатории, такой как ее название, предлагаемые услуги и т.д. Эта страница может иметь тег <header> для верхнего колонтитула, <nav> для панели навигации, <main> для основного содержимого и <footer> для нижнего колонтитула.
	\item Страница услуг: иметь список услуг, которые предлагает лаборатория. Каждая служба иметь свой собственный раздел с соответствующей формой. Эта страница будет иметь тег заголовка (<h1>) для заголовка, а затем использовать теги списка (<ul> и <li>) для представления предлагаемых услуг.
\end{enumerate}

\subsubsection{Особенности языка HTML}

Функции в смысле языковых функций не применяются в HTML, поскольку это только язык разметки, используемый для определения структуры и содержания веб-страницы. В HTML теги используются для определения таких элементов, как заголовки, абзацы, ссылки, списки и др. Эти теги не выполняют языковых функций, а указывают браузеру, как отображать и организовывать содержимое страницы. Однако в сочетании с другими языками функции могут быть использованы для обеспечения более динамичной функциональности веб-страницы.

HTML позволяет веб-разработчикам создавать веб-страницы и взаимодействовать с содержимым. К основным особенностям этого языка относятся:

\begin{itemize}
	\item Семантическая структура: HTML использует теги разметки для определения семантической структуры контента. Эти теги указывают, какой тип содержимого отображается, например, текст, изображения, гиперссылки, формы и т.д.
	\item Доступность: Для веб-разработки важно, чтобы ее содержание было доступно для всех пользователей, и HTML предоставляет дополнительные теги, позволяющие адаптировать содержание для обеспечения доступности для людей с ограниченными возможностями.
	\item Интерактивность: HTML позволяет создавать интерактивные формы, кнопки и другие элементы, которые позволяют пользователям взаимодействовать с содержимым веб-страницы.
	\item Совместимость с браузерами: HTML - это язык с высоким уровнем кросс-браузерной совместимости, обеспечивающий видимость содержимого веб-страницы на различных платформах.
\end{itemize}

\subsubsection{Недостатки использования HTML}

Вполне возможно, что использование HTML на веб-странице для медицинской лаборатории может иметь некоторые недостатки, хотя это в значительной степени зависит от контекста использования HTML. Некоторые возможные недостатки могут быть следующими:

\begin{enumerate}
	\item Ограничения в плане функциональности: HTML - это язык разметки, что означает, что он предназначен для создания статического контента в Интернете. Если на сайте требуется более продвинутая функциональность, например, интеграция с системами управления заказами или базами данных пациентов других клиник, может потребоваться использование других языков программирования.
	\item Проблемы совместимости с браузерами: HTML может иметь проблемы совместимости с некоторыми веб-браузерами. Если веб-страница не совместима с определенными браузерами, это может помешать врачам получить доступ и использовать информацию на странице.
	\item Отсутствие возможности настройки: Хотя HTML является очень гибким языком, могут существовать ограничения в настройке определенных аспектов веб-страницы, которые могут быть важны для конкретной медицинской лаборатории, например, включение определенных таблиц или графиков, специфичных для представления медицинских результатов.
\end{enumerate}

Ниже перечислены некоторые дополнительные недостатки языка разметки HTML:

\begin{itemize}
	\item Синтаксис HTML может быть сложным и не всегда легко читается или понимается пользователями, незнакомыми с этим языком.
	\item Теги HTML могут быть ограниченными и не всегда позволяют полностью и семантически описать информацию, которая должна быть отображена на странице.
	\item HTML не предлагает собственного способа работы с базами данных или сессиями, что может ограничить некоторые расширенные функциональные возможности веб-страницы.
	\item Разработка веб-страницы на HTML может быть более медленной и трудоемкой, чем на других более современных языках.
\end{itemize}

В целом, при разработке веб-сайта и определении наилучших инструментов для удовлетворения этих потребностей было уделено пристальное внимание конкретным потребностям медицинской лаборатории.

\subsection{Использование информационных технологий}

В настоящее время присутствие в Интернете необходимо для любого бизнеса, в том числе и для медицинских лабораторий. Хорошо продуманный и функциональный веб-сайт может стать ценным инструментом для улучшения коммуникации с пациентами и оптимизации внутренних лабораторных процессов.

Дизайн сайта медицинской лаборатории привлекателен и удобен для пользователей. Врачи могут легко найти необходимую им информацию, например, сведения о пациенте, историю болезни и т.д.. Кроме того, веб-сайт должен быть совместим с устройствами, используемыми на месте.

Дополнительной полезной технологией является возможность просмотра результатов анализов в режиме онлайн. Врачи могут получить доступ к результатам с любого компьютера с подключением к Интернету, что позволяет им принимать обоснованные решения относительно здоровья пациента.

Информационная безопасность имеет решающее значение в любом бизнесе, особенно в секторе здравоохранения. Веб-сайт должен быть защищен адекватными мерами безопасности. Кроме того.

Использование информационных технологий на веб-сайте медицинской лаборатории может значительно повысить эффективность и удовлетворенность пациентов и врачей.