\newsection
\centertocsection{ВВЕДЕНИЕ}

В 1989 году Тим Бернерс-Ли, ученый-компьютерщик из ЦЕРН в Швейцарии, разработал концепцию и технологию того, что в итоге стало Всемирной паутиной. Первая веб-страница была опубликована в 1991 году, и в последующие годы веб-технологии быстро развивались, позволяя создавать динамичные и привлекательные веб-сайты. Сегодня Интернет является основополагающим компонентом повседневной жизни и вносит большой вклад в мировую экономику.

HTML необходим для создания веб-страниц, но он также дополняется другими языками программирования, такими как CSS (каскадные таблицы стилей) для визуального представления веб-страниц и JavaScript для добавления интерактивности веб-страницам.

Веб-страница значительно эволюционировала с первых дней существования Интернета до наших дней, во многом благодаря языку HTML. По мере развития технологий HTML развивался, позволяя создавать более интерактивные и визуально привлекательные веб-страницы. В прошлом HTML использовался в основном для создания базовых веб-страниц без большого количества графических или интерактивных ресурсов. Но теперь HTML развился и позволяет создавать анимацию, видео, интерактивные элементы и многое другое.

Использование веб-сайтов в медицинских лабораториях необходимо для распространения информации и поддержания более эффективной связи с пользователями, что, в свою очередь, может привести к повышению качества предлагаемых услуг. Веб-сайты могут использоваться для продвижения услуг, предлагаемых лабораториями, а также для информирования о процедурах, мерах безопасности и других важных аспектах. Кроме того, они могут облегчить доступ к результатам анализов и проведенных тестов, что может быть очень полезно для лечащих врачей. Эффективное использование веб-сайтов для повышения качества услуг и удовлетворенности пользователей.

\emph{Цель настоящей работы} – Разработать сайт медицинской лаборатории для эффективной регистрации информации о пациентах, увеличения количества пациентов и обнародования результатов медицинских обследований. Для достижения поставленной цели необходимо решить следующие задачи \emph{следующие задачи:}
\begin{itemize}
\item провести анализ предметной области;
\item разработать концептуальную модель web-сайта;
\item спроектировать web-сайт;
\item реализовать сайт средствами web-технологий.
\end{itemize}

\emph{Структура и объем работы.} Отчет состоит из введения, 4 разделов основной части, заключения, списка использованных источников, 2 приложений. Текст выпускной квалификационной работы равен \formbytotal{page}{страниц}{е}{ам}{ам}.

\emph{Во введении} сформулирована цель работы, поставлены задачи разработки, описана структура работы, приведено краткое содержание каждого из разделов.

\emph{В первом разделе} на стадии описания технической характеристики предметной области приводится сбор информации о медицинской лаборатории, для которой ведется разработка сайта.

\emph{Во втором разделе} на стадии технического задания приводятся требования к разрабатываемому сайту.

\emph{В третьем разделе} на стадии технического проектирования представлены проектные решения для web-сайта.

\emph{В четвертом разделе} приводится список классов и их методов, использованных при разработке сайта, производится тестирование разработанного сайта.

В заключении излагаются основные результаты работы, полученные в ходе разработки.

В приложении А представлен графический материал.
В приложении Б представлены фрагменты исходного кода. 
